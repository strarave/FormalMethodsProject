\documentclass[10pt,a4paper]{article}

\usepackage{enumitem}

\title{Stochastic Model Checking of Warehouse Robotics}
\author{Corti Simone, Ravella Elia, Sarneri Enrico}
\begin{document}
	\begin{titlepage}
		\maketitle
	\end{titlepage}
	
	\section{The Model}
		\subsection{Grid}
			The grid is implemented as a global bidimensional array of integers. This implementation allowed us to manage in a detailed way the single cells, and also helped us simplify the movement of the bots. We did not insert a template Grid because this would have required the introduction of many additional synchronization channels and would have been very difficult to integrate with other templates.
			The integers that model the grid cells \emph{bounded}: they can only assume 8 different values. These are
			\begin{enumerate}[start=0, label={\arabic* :}]
				\item the cell is free, so no bot or pods are in that cell
				\item the cell is occupied only by a bot
				\item the cell is occupied only by a pod that is not claimed
				\item the cell is occupied only by a pod that is claimed by a task 
				\item the cell is the entry point 
				\item the cell is the delivery point
				\item the cell is occupied by a bot and a free pod
				\item the cell is occupied by a bot and a claimed pod
			\end{enumerate}
			All these slight differences among cell states are needed to safely manage the movement of the bots \emph{and} the assignement of a pod to a task. Moreover, adding two cell state to the grid representation allowed us to completely remove the "Pod" template, hence making the model slimmer.\\
			The grid is highly configurable; these are the parameters that can be changed:
			\begin{itemize}
				\item lenght of the grid
				\item width of the grid
				\item entry point position coloumn (can be moved along the lower edge of the grid but cannot enter the last 3 coloumns) 
				\item delivery point position coloumn (can be moved in the last three)
				\item row lenght
				\item space between rows
			\end{itemize}
			Unfortunately, Uppaal requires this parameters to be constant at compile time, so they have to be manually changed in the template. We would have preferred to parametrize them, so to have to modify only the global definition. 
			
		
		\subsection{Bot}
			The Bot template has the task to model the behaviour of the various robots in the system. \\
			It accepts as input parameter the custom-type variable $bot_t$, which represents the ID assigned to the bot.\\
			The template is compesed of the following states:
			\begin{itemize}
				\item \textbf{\underline{Idle}}: this is the initial state of the template and it models the bot when it has no claimed task.\\The bot may leave this state when the number of available tasks is greater than zero. Furthermore, the probability that the bot is leaving the states is modeled as an exponential distribution with $\chi\;\sim\;(\lambda)$;
				\item \textbf{\underline{Claimed}}: this state represents the bot after it claimed a task from the queue until it reaches the delivery point. The bot is forced to leave the state after \emph{K} seconds;
				\item \textbf{\underline{Delivered}}: the bot reaches this state when its position coincides whit the delivery point. The bot stays in the state until it recevies a \emph{PickUp} message from the \emph{Human};
				\item \textbf{\underline{Returning}}: after leaving the delivery point, the bot returns back to the \emph{pod position} and, after this, it goes back to the \emph{entry point}. As in claimed, the bot must leave the state after \emph{K} seconds.
			\end{itemize}
			When the pod is moving to reach its destination, it follows a predetermined path wihch will be explained in the following section.
			\subsubsection{Movement}
				As we can see from the template, the movement of the robot is controlled by a function called \emph{chooseStrategy}.\\
				This function uses a parameter called \emph{strategy} to determine in which part of the path the robot is and to apply the corresponding movement function.\\
				\begin{itemize}
					\item \textbf{\underline{From entry point to pod position}}
				\end{itemize}
		
		\subsection{Queue}
			The Queue template models the mechanism to manage the arriving of new tasks in the system and their dispatching to the bots. It also carries out the grid initialization.\\
			It is composed of two states:
			\begin{itemize}
				\item An initial \emph{start} state that we made also committed to carry out the queue and grid initializations; originally, the setting of the grid were carried out by a dedicated template, but since it was the only operation in that process we removed it and integrated in this one in order to simplify the model.
				\item A \emph{working} state, that groups the functions of
					\begin{enumerate}
						\item Adding a task to the queue
						\item Removing a task from the queue when it's been claimed by a bot
						\item Keeping count of how many tasks has been issued, how many tasks are executed and how many tasks have been lost
					\end{enumerate}
					This last function is crucial, because the property that must hold in the system throughout the whole test run will directly address the number of tasks that have been lost managed by this process.
			\end{itemize}
			We have initally modeled this template with a state dedicated to each one of the function listed above, then we progressively removed the states "embedding" their functionalities in the transitions and grouping them together. This was done in order to reduce the number of states to verify and improve the verification time. 
		
		\subsection{Task and Human}
			The Task template models the generation process of new tasks. Its parameters sets the values of mean and standard deviation of the normal distribution associated to the "spawn rate" of new tasks.\\
			It's composed of two states:
			\begin{itemize}
				\item A \emph{startingTask} initial commit state that is in charge of initializing the additional data structures of the model, such as the list of available pods.
				\item An \emph{idle} state that just sends tokens on the channel \emph{newTask} at random interval of time (according to the distribution specified).
			\end{itemize}
			
	
	\section{System Configurations}
	
	\section{Experimental Results}
	
	\section{Further Improvement and Alternative Approaches}
	
	
\end{document}
